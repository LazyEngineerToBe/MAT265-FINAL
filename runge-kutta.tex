\section{Algorithme de Runge-Kutta}
\raggedright
\subsection{$1^{\mathrm{er}}$ ordre}

\begin{equation*}
    \left\{ \begin{array}{rcl}
        y^{\prime} & = & x^2+y^2 \\
        y(x_0) & = & y_0
    \end{array}\right.
\end{equation*}

\subsubsection{TI}
\menu[,]{Menu,3,8}
\begin{equation*}
    \left\{ \begin{array}{rcl}
        y1^{\prime} & = & x^2+y1^2 \\
        (x_0,y1_0) & = & y_0
    \end{array}\right.
\end{equation*}

Puis: \menu[,]{\ldots,Runge-Kutta, Tol. erreur: 0.0001, Champ: Aucun}

\subsection{$2^{\mathrm{e}}$ ordre}

\begin{equation*}
    \left\{ \begin{array}{rcl}
        (x-5)y^{\prime\prime} +xy^{\prime}+3y & = & 0 \\
        y(x_0) & = & y_0\\
        y^{\prime}(x_0) & = & y^{\prime}_0
    \end{array}\right.
\end{equation*}

%\raggedright

On transforme l'É.D. en É.D. du $1^{\mathrm{er}}$ ordre.\\
On pose: \( y=y_1 \;\mathrm{ et }\; y^{\prime}=y_2 \Longleftrightarrow y^{\prime\prime}=y^{\prime}_2 \)\\
\subsubsection{TI}
\centering
\(\mathbf{de2to1}(Py^{\prime\prime} +Qy^{\prime}+Ry = 0)\)\\\raggedright

\menu[,]{Menu,3,8}
\begin{equation*}
    \left\{ \begin{array}{rcl}
        y1^{\prime} & = & y2 \\
        (x_0,y1_0) & = & y_0\\
        y2^{\prime} & = & y_2^{\prime} \quad\mathrm{(E.D.}\ 1^{er}\ \mathrm{ ordre)}\\
        (x_0, y2_0) & = & y^{\prime}_0
    \end{array}\right.
\end{equation*}

Puis: \menu[,]{\ldots,Runge-Kutta, Tol. erreur: 0.0001, Champ: Aucun}

\subsection{Éditer les valeurs dans le tableau}

\begin{tabular}{cl}
\keys{ctrl+T} & Afficher/Quitter la table des valeurs\\
\keys{menu+2+5} & Paramètres de la table
\end{tabular}