\subsection{Circuit RLC}
{\centering
\resizebox{.1\textwidth}{!}{
\begin{circuitikz}
    \draw (0,0)
    to[R, l=$R$] (2,0)
    to[L, l=$L$] (2,-2)
    to[C, l=$C$] (0,-2)
    to[V, l=$V$, invert] (0,0);
\end{circuitikz}
}
}
\\\raggedright
Voltage au capaciteur:
{\centering
%\begin{gather*}
\(
LC V_c^{\prime\prime} + RC V_c^{\prime} + V_c = V(t)\Longleftrightarrow
V_c^{\prime\prime} + \frac{R}{L} V_c^{\prime} + \frac{1}{LC}V_c = \frac{V(t)}{LC}
\)}\\
%\end{gather*}
Courant:
{\centering\(
%\begin{gather*}
L i^{\prime\prime} + R i^{\prime} + \frac{1}{C}i = E(t)\;\;\Longleftrightarrow\;\;
i^{\prime\prime} + \frac{R}{L} i^{\prime} + \frac{1}{LC}i = \frac{E(t)}{L}
% \end{gather*}
\)}\\
Avec:
\begin{tabular}{rlc}
  $R$ & résistance & [$\Omega$] \\
  $L$ & inductance & [$H$]\\
  $C$ & capacité & [$F$]\\
  $E(t)=V(t)$ & source de puissance & [$V$]\\
  $V_c(t)*C = i(t_0)$ & courant à $t_0$ & [$A$]\\
  $V_c(t_0)=i^{\prime}(t_0)$ & voltage au capaciteur à $t_0$ & [$V$]
\end{tabular}


\subsubsection{TI}
\centering
\(\mathbf{circuit}(R,L,C,E,Vc_0,I_0)\)\\