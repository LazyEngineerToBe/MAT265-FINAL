

\section{Mouvement rectiligne}
\vspace{-2\baselineskip}
\begin{equation*}
m\frac{dv}{dt}=\pm \:mg-kv \;\mid k>0,\;v(0)=\;\textrm{vitesse initiale}
\end{equation*}
\begin{align*}
    a(t) &= \frac{dv}{dt} = \frac{d^2x}{d^2t} = at^2 +a_0\\
    v(t) &= \frac{dx}{dt} = vt+v_0\\
    x(t) &= x_0 + v t + \frac{1}{2}a t^2
\end{align*}

\subsection{Vitesse limite théorique}
\begin{equation*}
v_\infty = \lim_{t\rightarrow 0} v(t)
\end{equation*}

\section{Circuits électriques}
\vspace{-2\baselineskip}
\subsection{Circuit RC}
\[RC\frac{dv_c}{dt}+v_c = v \;\mid v_c(0) = \textrm{tension initial du condensateur}\]

Ensuite, on trouve le courant électrique dans le circuit avec la relation:
\(i(t)=C\frac{dv_c}{dt}\)


\subsection{Circuit RL}
\[L\frac{di}{dt}+Ri=V \;\mid i(0) = \textrm{courant initial souvent}\; i(0)=0\]

\section{Loi de refroidissement Newton}
\vspace{-2\baselineskip} 
\begin{align*}
    \frac{dT}{dt} &= k(T-T_A) \\
    T &= T_A + Ce^{kt}    
\end{align*}
$T$: Température objet, $T_A$: Temp. ambiante, $k$: Constante



\section{Mélanges}
\vspace{-2\baselineskip}
\[ \frac{dq}{dt}=T_e - T_s \;\mid \textrm{où}\; T_e = \textrm{taux à l'entrée et}\; T_s = \textrm{taux à la sortie}\]
